
%% FinalProject.tex
%% CS698 - Spring 2012
%% Professor: Amarda Shehu
%%
%% Team: Brian Moriarty / Stuart Roettger
%% May 11, 2012
%%
%% This is the Latex code for our final project, formatting based on the
%% IEEE bare_adv.tex template provided by Michael Shell.
%%

\documentclass[12pt,journal,compsoc]{IEEEtran}
% The Computer Society requires 12pt.

% *** GRAPHICS RELATED PACKAGES ***
%
\ifCLASSINFOpdf
   \usepackage[pdftex]{graphicx}
  % declare the path(s) where your graphic files are
   \graphicspath{{../pdf/}{../jpeg/}}
  % and their extensions so you won't have to specify these with
  % every instance of \includegraphics
   \DeclareGraphicsExtensions{.pdf,.jpeg,.png,.jpg}
\else
  % or other class option (dvipsone, dvipdf, if not using dvips). graphicx
  % will default to the driver specified in the system graphics.cfg if no
  % driver is specified.
  % \usepackage[dvips]{graphicx}
  % declare the path(s) where your graphic files are
  % \graphicspath{{../eps/}}
  % and their extensions so you won't have to specify these with
  % every instance of \includegraphics
  % \DeclareGraphicsExtensions{.eps}
\fi

\providecommand{\PSforPDF}[1]{#1}
% Note that in order for ps4pdf to work, all commands related to psfrag,
% pstricks, etc. must be called within the PSforPDF command. This applies
% even when *loading* via \usepackage psfrag.sty, etc.

%\usepackage{algorithmic}
\usepackage{verbatim}
\usepackage{algpseudocode}
\usepackage{algorithm}
%
% TODO: the IEEE coments below warn not to use algorithm, but instead
% to use algorithmic... however, algorithmic either looks terrible or
% at the very least does not explain how to make its format look good,
% so I am using the algorithm package anyway for now.
% I am not sure if this will break the typesetting later, but it looks
% good so far.
%

% algorithmic.sty was written by Peter Williams and Rogerio Brito.
% This package provides an algorithmic environment fo describing algorithms.
% You can use the algorithmic environment in-text or within a figure
% environment to provide for a floating algorithm. Do NOT use the algorithm
% floating environment provided by algorithm.sty (by the same authors) or
% algorithm2e.sty (by Christophe Fiorio) as IEEE does not use dedicated
% algorithm float types and packages that provide these will not provide
% correct IEEE style captions. The latest version and documentation of
% algorithmic.sty can be obtained at:
% http://www.ctan.org/tex-archive/macros/latex/contrib/algorithms/
% There is also a support site at:
% http://algorithms.berlios.de/index.html
% Also of interest may be the (relatively newer and more customizable)
% algorithmicx.sty package by Szasz Janos:
% http://www.ctan.org/tex-archive/macros/latex/contrib/algorithmicx/




% *** ALIGNMENT PACKAGES ***
%
%\usepackage{array}
% Frank Mittelbach's and David Carlisle's array.sty patches and improves
% the standard LaTeX2e array and tabular environments to provide better
% appearance and additional user controls. As the default LaTeX2e table
% generation code is lacking to the point of almost being broken with
% respect to the quality of the end results, all users are strongly
% advised to use an enhanced (at the very least that provided by array.sty)
% set of table tools. array.sty is already installed on most systems. The
% latest version and documentation can be obtained at:
% http://www.ctan.org/tex-archive/macros/latex/required/tools/


%\usepackage{mdwmath}
%\usepackage{mdwtab}
% Also highly recommended is Mark Wooding's extremely powerful MDW tools,
% especially mdwmath.sty and mdwtab.sty which are used to format equations
% and tables, respectively. The MDWtools set is already installed on most
% LaTeX systems. The lastest version and documentation is available at:
% http://www.ctan.org/tex-archive/macros/latex/contrib/mdwtools/


% IEEEtran contains the IEEEeqnarray family of commands that can be used to
% generate multiline equations as well as matrices, tables, etc., of high
% quality.


%\usepackage{eqparbox}
% Also of notable interest is Scott Pakin's eqparbox package for creating
% (automatically sized) equal width boxes - aka "natural width parboxes".
% Available at:
% http://www.ctan.org/tex-archive/macros/latex/contrib/eqparbox/



% *** SUBFIGURE PACKAGES ***
%\ifCLASSOPTIONcompsoc
%\usepackage[tight,normalsize,sf,SF]{subfigure}
%\else
%\usepackage[tight,footnotesize]{subfigure}
%\fi
% subfigure.sty was written by Steven Douglas Cochran. This package makes it
% easy to put subfigures in your figures. e.g., "Figure 1a and 1b". For IEEE
% work, it is a good idea to load it with the tight package option to reduce
% the amount of white space around the subfigures. Computer Society papers
% use a larger font and \sffamily font for their captions, hence the
% additional options needed under compsoc mode. subfigure.sty is already
% installed on most LaTeX systems. The latest version and documentation can
% be obtained at:
% http://www.ctan.org/tex-archive/obsolete/macros/latex/contrib/subfigure/
% subfigure.sty has been superceeded by subfig.sty.

\usepackage{caption}
\usepackage{subcaption}
%\ifCLASSOPTIONcompsoc
%  \usepackage[caption=false]{caption}
%  \usepackage[font=normalsize,labelfont=sf,textfont=sf]{subfig}
%\else
%  \usepackage[caption=false]{caption}
%  \usepackage[font=footnotesize]{subfig}
%\fi
% subfig.sty, also written by Steven Douglas Cochran, is the modern
% replacement for subfigure.sty. However, subfig.sty requires and
% automatically loads Axel Sommerfeldt's caption.sty which will override
% IEEEtran.cls handling of captions and this will result in nonIEEE style
% figure/table captions. To prevent this problem, be sure and preload
% caption.sty with its "caption=false" package option. This is will preserve
% IEEEtran.cls handing of captions. Version 1.3 (2005/06/28) and later 
% (recommended due to many improvements over 1.2) of subfig.sty supports
% the caption=false option directly:
%\ifCLASSOPTIONcompsoc
%  \usepackage[caption=false,font=normalsize,labelfont=sf,textfont=sf]{subfig}
%\else
%  \usepackage[caption=false,font=footnotesize]{subfig}
%\fi
%
% The latest version and documentation can be obtained at:
% http://www.ctan.org/tex-archive/macros/latex/contrib/subfig/
% The latest version and documentation of caption.sty can be obtained at:
% http://www.ctan.org/tex-archive/macros/latex/contrib/caption/




% *** FLOAT PACKAGES ***
%
%\usepackage{fixltx2e}
% fixltx2e, the successor to the earlier fix2col.sty, was written by
% Frank Mittelbach and David Carlisle. This package corrects a few problems
% in the LaTeX2e kernel, the most notable of which is that in current
% LaTeX2e releases, the ordering of single and double column floats is not
% guaranteed to be preserved. Thus, an unpatched LaTeX2e can allow a
% single column figure to be placed prior to an earlier double column
% figure. The latest version and documentation can be found at:
% http://www.ctan.org/tex-archive/macros/latex/base/


%\usepackage{stfloats}
% stfloats.sty was written by Sigitas Tolusis. This package gives LaTeX2e
% the ability to do double column floats at the bottom of the page as well
% as the top. (e.g., "\begin{figure*}[!b]" is not normally possible in
% LaTeX2e). It also provides a command:
%\fnbelowfloat
% to enable the placement of footnotes below bottom floats (the standard
% LaTeX2e kernel puts them above bottom floats). This is an invasive package
% which rewrites many portions of the LaTeX2e float routines. It may not work
% with other packages that modify the LaTeX2e float routines. The latest
% version and documentation can be obtained at:
% http://www.ctan.org/tex-archive/macros/latex/contrib/sttools/
% Documentation is contained in the stfloats.sty comments as well as in the
% presfull.pdf file. Do not use the stfloats baselinefloat ability as IEEE
% does not allow \baselineskip to stretch. Authors submitting work to the
% IEEE should note that IEEE rarely uses double column equations and
% that authors should try to avoid such use. Do not be tempted to use the
% cuted.sty or midfloat.sty packages (also by Sigitas Tolusis) as IEEE does
% not format its papers in such ways.


%\ifCLASSOPTIONcaptionsoff
%  \usepackage[nomarkers]{endfloat}
% \let\MYoriglatexcaption\caption
% \renewcommand{\caption}[2][\relax]{\MYoriglatexcaption[#2]{#2}}
%\fi
% endfloat.sty was written by James Darrell McCauley and Jeff Goldberg.
% This package may be useful when used in conjunction with IEEEtran.cls'
% captionsoff option. Some IEEE journals/societies require that submissions
% have lists of figures/tables at the end of the paper and that
% figures/tables without any captions are placed on a page by themselves at
% the end of the document. If needed, the draftcls IEEEtran class option or
% \CLASSINPUTbaselinestretch interface can be used to increase the line
% spacing as well. Be sure and use the nomarkers option of endfloat to
% prevent endfloat from "marking" where the figures would have been placed
% in the text. The two hack lines of code above are a slight modification of
% that suggested by in the endfloat docs (section 8.3.1) to ensure that
% the full captions always appear in the list of figures/tables - even if
% the user used the short optional argument of \caption[]{}.
% IEEE papers do not typically make use of \caption[]'s optional argument,
% so this should not be an issue. A similar trick can be used to disable
% captions of packages such as subfig.sty that lack options to turn off
% the subcaptions:
% For subfig.sty:
% \let\MYorigsubfloat\subfloat
% \renewcommand{\subfloat}[2][\relax]{\MYorigsubfloat[]{#2}}
% For subfigure.sty:
% \let\MYorigsubfigure\subfigure
% \renewcommand{\subfigure}[2][\relax]{\MYorigsubfigure[]{#2}}
% However, the above trick will not work if both optional arguments of
% the \subfloat/subfig command are used. Furthermore, there needs to be a
% description of each subfigure *somewhere* and endfloat does not add
% subfigure captions to its list of figures. Thus, the best approach is to
% avoid the use of subfigure captions (many IEEE journals avoid them anyway)
% and instead reference/explain all the subfigures within the main caption.
% The latest version of endfloat.sty and its documentation can obtained at:
% http://www.ctan.org/tex-archive/macros/latex/contrib/endfloat/
%
% The IEEEtran \ifCLASSOPTIONcaptionsoff conditional can also be used
% later in the document, say, to conditionally put the References on a 
% page by themselves.





% *** PDF, URL AND HYPERLINK PACKAGES ***
%
%\usepackage{url}
% url.sty was written by Donald Arseneau. It provides better support for
% handling and breaking URLs. url.sty is already installed on most LaTeX
% systems. The latest version can be obtained at:
% http://www.ctan.org/tex-archive/macros/latex/contrib/misc/
% Read the url.sty source comments for usage information. Basically,
% \url{my_url_here}.


% NOTE: PDF thumbnail features are not required in IEEE papers
%       and their use requires extra complexity and work.
%\ifCLASSINFOpdf
%  \usepackage[pdftex]{thumbpdf}
%\else
%  \usepackage[dvips]{thumbpdf}
%\fi
% thumbpdf.sty and its companion Perl utility were written by Heiko Oberdiek.
% It allows the user a way to produce PDF documents that contain fancy
% thumbnail images of each of the pages (which tools like acrobat reader can
% utilize). This is possible even when using dvi->ps->pdf workflow if the
% correct thumbpdf driver options are used. thumbpdf.sty incorporates the
% file containing the PDF thumbnail information (filename.tpm is used with
% dvips, filename.tpt is used with pdftex, where filename is the base name of
% your tex document) into the final ps or pdf output document. An external
% utility, the thumbpdf *Perl script* is needed to make these .tpm or .tpt
% thumbnail files from a .ps or .pdf version of the document (which obviously
% does not yet contain pdf thumbnails). Thus, one does a:
% 
% thumbpdf filename.pdf 
%
% to make a filename.tpt, and:
%
% thumbpdf --mode dvips filename.ps
%
% to make a filename.tpm which will then be loaded into the document by
% thumbpdf.sty the NEXT time the document is compiled (by pdflatex or
% latex->dvips->ps2pdf). Users must be careful to regenerate the .tpt and/or
% .tpm files if the main document changes and then to recompile the
% document to incorporate the revised thumbnails to ensure that thumbnails
% match the actual pages. It is easy to forget to do this!
% 
% Unix systems come with a Perl interpreter. However, MS Windows users
% will usually have to install a Perl interpreter so that the thumbpdf
% script can be run. The Ghostscript PS/PDF interpreter is also required.
% See the thumbpdf docs for details. The latest version and documentation
% can be obtained at.
% http://www.ctan.org/tex-archive/support/thumbpdf/
% Be sure and use only version 3.8 (2005/07/06) or later of thumbpdf as
% earlier versions will not work properly with recent versions of pdfTeX
% (1.20a and later).


% correct bad hyphenation here
\hyphenation{op-tical net-works semi-conduc-tor}


\begin{document}
%
% paper title
% can use linebreaks \\ within to get better formatting as desired
\title{\large Final Project - Planning Motions of \\ Multiple Robots with Prioritized Planning}
%
%
% author names and IEEE memberships
% note positions of commas and nonbreaking spaces ( ~ ) LaTeX will not break
% a structure at a ~ so this keeps an author's name from being broken across
% two lines.
% use \thanks{} to gain access to the first footnote area
% a separate \thanks must be used for each paragraph as LaTeX2e's \thanks
% was not built to handle multiple paragraphs
%
%
%\IEEEcompsocitemizethanks is a special \thanks that produces the bulleted
% lists the Computer Society journals use for "first footnote" author
% affiliations. Use \IEEEcompsocthanksitem which works much like \item
% for each affiliation group. When not in compsoc mode,
% \IEEEcompsocitemizethanks becomes like \thanks and
% \IEEEcompsocthanksitem becomes a line break with idention. This
% facilitates dual compilation, although admittedly the differences in the
% desired content of \author between the different types of papers makes a
% one-size-fits-all approach a daunting prospect. For instance, compsoc 
% journal papers have the author affiliations above the "Manuscript
% received ..."  text while in non-compsoc journals this is reversed. Sigh.

\author{\small Brian~Moriarty
        /~Stuart~Roettger% <-this % stops a space
\IEEEcompsocitemizethanks{\IEEEcompsocthanksitem B. Moriarty is a student at George Mason University.\protect\\
% note need leading \protect in front of \\ to get a newline within \thanks as
% \\ is fragile and will error, could use \hfil\break instead.
E-mail: bmoriart@gmu.edu
\IEEEcompsocthanksitem S. Roettger is a student at George Mason University.\protect\\
E-mail: sroettger@gmu.edu}% <-this % stops a space
\thanks{CS689 Project turned in May 11, 2012.}}

% note the % following the last \IEEEmembership and also \thanks - 
% these prevent an unwanted space from occurring between the last author name
% and the end of the author line. i.e., if you had this:
% 
% \author{....lastname \thanks{...} \thanks{...} }
%                     ^------------^------------^----Do not want these spaces!
%
% a space would be appended to the last name and could cause every name on that
% line to be shifted left slightly. This is one of those "LaTeX things". For
% instance, "\textbf{A} \textbf{B}" will typeset as "A B" not "AB". To get
% "AB" then you have to do: "\textbf{A}\textbf{B}"
% \thanks is no different in this regard, so shield the last } of each \thanks
% that ends a line with a % and do not let a space in before the next \thanks.
% Spaces after \IEEEmembership other than the last one are OK (and needed) as
% you are supposed to have spaces between the names. For what it is worth,
% this is a minor point as most people would not even notice if the said evil
% space somehow managed to creep in.



% The paper headers
%\markboth{Journal of \LaTeX\ Class Files,~Vol.~6, No.~1, January~2007}%
%{Shell \MakeLowercase{\textit{et al.}}: Bare Advanced Demo of IEEEtran.cls for Journals}
% The only time the second header will appear is for the odd numbered pages
% after the title page when using the twoside option.
% 
% *** Note that you probably will NOT want to include the author's ***
% *** name in the headers of peer review papers.                   ***
% You can use \ifCLASSOPTIONpeerreview for conditional compilation here if
% you desire.



% The publisher's ID mark at the bottom of the page is less important with
% Computer Society journal papers as those publications place the marks
% outside of the main text columns and, therefore, unlike regular IEEE
% journals, the available text space is not reduced by their presence.
% If you want to put a publisher's ID mark on the page you can do it like
% this:
%\IEEEpubid{0000--0000/00\$00.00~\copyright~2007 IEEE}
% or like this to get the Computer Society new two part style.
%\IEEEpubid{\makebox[\columnwidth]{\hfill 0000--0000/00/\$00.00~\copyright~2007 IEEE}%
%\hspace{\columnsep}\makebox[\columnwidth]{Published by the IEEE Computer Society\hfill}}
% Remember, if you use this you must call \IEEEpubidadjcol in the second
% column for its text to clear the IEEEpubid mark (Computer Society jorunal
% papers don't need this extra clearance.)



% use for special paper notices
%\IEEEspecialpapernotice{(Invited Paper)}



% for Computer Society papers, we must declare the abstract and index terms
% PRIOR to the title within the \IEEEcompsoctitleabstractindextext IEEEtran
% command as these need to go into the title area created by \maketitle.
%\IEEEcompsoctitleabstractindextext{%
%\begin{abstract}
%\boldmath
%The abstract goes here.
%\end{abstract}
% IEEEtran.cls defaults to using nonbold math in the Abstract.
% This preserves the distinction between vectors and scalars. However,
% if the journal you are submitting to favors bold math in the abstract,
% then you can use LaTeX's standard command \boldmath at the very start
% of the abstract to achieve this. Many IEEE journals frown on math
% in the abstract anyway. In particular, the Computer Society does
% not want either math or citations to appear in the abstract.

% Note that keywords are not normally used for peerreview papers.
%\begin{IEEEkeywords}
%GMU, CS689, \LaTeX, paper, Motion, Planning.
%\end{IEEEkeywords}}


% make the title area
\maketitle


% To allow for easy dual compilation without having to reenter the
% abstract/keywords data, the \IEEEcompsoctitleabstractindextext text will
% not be used in maketitle, but will appear (i.e., to be "transported")
% here as \IEEEdisplaynotcompsoctitleabstractindextext when compsoc mode
% is not selected <OR> if conference mode is selected - because compsoc
% conference papers position the abstract like regular (non-compsoc)
% papers do!
\IEEEdisplaynotcompsoctitleabstractindextext
% \IEEEdisplaynotcompsoctitleabstractindextext has no effect when using
% compsoc under a non-conference mode.


% For peer review papers, you can put extra information on the cover
% page as needed:
% \ifCLASSOPTIONpeerreview
% \begin{center} \bfseries EDICS Category: 3-BBND \end{center}
% \fi
%
% For peerreview papers, this IEEEtran command inserts a page break and
% creates the second title. It will be ignored for other modes.
\IEEEpeerreviewmaketitle


%%%%%%%%%%%%%%%%%%%%%%%%%%%%%%%%%%%%%%%%%%%%%%%%%%%%%%%%%%%%%
\section{Introduction}
\IEEEPARstart{P}{lanning} motions of multiple robots or of robots in dynamic environments presents new challenges in the form of coordination which are not encountered by single robot path planners in static environments. Planning algorithms which work well for single robots in static environments are not necessarily effective in dynamic scenes and must therefore be reevaluated in the context of dynamic multi-robot scenes. Prioritized Planning is an approach to multiple robot planning which permits each robot to be associated with its own unique starting state and goal state and then planned individually using any planner that could be used in a single robot environment. Our work investigates how the prioritized order of the robots impacts whether all robots are able to achieve their goal state. Our experiments suggest that while the priority function is an important component in some scenes, it is unlikely that any heuristic based priority function exists that will lead to success for all scenes. We further demonstrate that the choice of individual algorithms for each robot can have a significant impact on the solvability of the scene.
\par
Section 2 discusses single robot algorithms and the intent of prioritized planning. Section 3 describes our implementation of prioritized planning. Section 4 shows experimental scenes with and without using our priority heuristic, and also demonstrates the impact of alternative single robot planners on the same scenes. Section 5 discusses the efforts of other researchers on this problem, and describes how their proposed solutions improve the chance of success using methods other than prioritized planning.

%%%%%%%%%%%%%%%%%%%%%%%%%%%%%%%%%%%%%%%%%%%%%%%%%%%%%%%%%%%%%
\section{Background}
\emph{Centralized Planning} is a strategy for solving movements of multiple robots by treating the group as a having a single shared state. Each sample state includes the configuration for all of the robots, and one robot in an invalid configuration can invalidate the entire sample. This approach is effective up to a point, but it does not scale well. Increasing the number of robots has the same effect as increasing the dimensionality of the problem \cite{lavalle}.
\par
\emph{Decoupled Planning} is a strategy for solving multiple robot problems where each robot can be solved for separately, and then various sub-categories of decoupled planning describe strategies for coordinating the resulting individual paths. One of the benefits of the decoupled approach is that scaling in the number of robots is just like solving for each robot one after the other, so this approach inherits the scaling characteristics of the underlying single robot planner. 
\par
\emph{Prioritized Planning} is often considered a sub-category of Decoupled Planning \cite{lavalle}. In prioritized planning, a priority order is set for a group of robots ${A_{i..n}}$, ahead of the planning process. Then, as each robot ${A_{i}}$ discovers its own path to the goal, the next ${A_{i+1}}$ can solve for its path treating all ${A_{1..i}}$ robots as obstacles. The key to this approach is being able to easily identify where any ${A_{i}}$ robot plans to be at a given time step so collisions can be detected accurately.

%%%%%%%%%%%%%%%%%%%%%%%%%%%%%%%%%%%%%%%%%%%%%%%%%%%%%%%%%%%%%
\section{Approach}
\subsection{Simulation Framework}
We modified the Homework 3 framework to support multiple robots. In Homework 3 a scene was comprised of a single disc robot, a single disc goal and multiple disc obstacles. Our modified framework continues to use disc obstacles, but now supports multiple rectangular robots. Each robot is associated with its own disc goal. A robot geometry is defined by length and width. The robot state is defined by a center point and orientation. In addition, each robot may be associated with a planner implementation specific to it,  offering support for testing how different planning algorithms interact with each other. Lastly, the file format allows for identifying the priority function to be used when deciding the robot planning order.

\subsection{Algorithm}
We follow the basic algorithm for prioritized planning as listed in Algorithm \ref{priplanner}. The priority of the robot planning order is set based on a user specified priority function. We use the priority function to build a max heap, with the higher priority robots being popped off and solved one at a time. Line 8 of the algorithm is calling solve on the current robot and indicates the robot should avoid previously solved robots, in addition to any obstacles.
\par
We provide two categories of priority function. The baseline prioritizes the planners in either forward or reverse order of how they are listed in the configuration file. The second option prioritize the planners according to their minimum average obstacle distance. Our implementation of this second approach figures out the priorities at runtime, but does not change the priorities once the planning process has begun. This differs from the \emph{Dynamic Priority Scheme} described in \cite{slides} where priorities may be reordered periodically at meaningful time steps. We chose to use \emph{Static Priority Schemes} where lower priority robots are not considered until higher priority robots have a full path to the goal. We made this decision for 2 reasons. First, although a dynamic approach would offer additional opportunities to coordinate movement, our scenes expose initial states where priority alone does not produce a different outcome. Second, the expense of reprioritizing and replanning each robot seemed unjustified in our scenes where the robots are the only mobile objects. We think \emph{Dynamic Prioritization} would be more valuable in scenes where objects may move outside of what the robots are coordinating among themselves.

%%%%%%%%%%%%%%%%%%%
% pdeudo code
\begin{algorithm}
\caption{Prioritized Planning}\label{priplanner}
\begin{algorithmic}[1]
\State $M \gets$ list of N new Robots
\State $Heap \gets$ $MaxHeap(M)_{priority function}$
\While{not $Heap.empty$}
	\State $A \gets Heap.pop$
	\State $A.solve(avoid$[$obstacles + M_{solved==true}$]$)$
	\State $A.solved = true$
\EndWhile
\end{algorithmic}
\end{algorithm}
%%%%%%%%%%%%%%%%%%%

The solve function on line 8 is where any arbitrary single robot planning algorithm may be used. When a random state is chosen for $A$ and checked against the other $M_{solved=true}$ robots, a timing function is required in order to predict where the other robots will be at the time when $A$ would be at the random state. Algorithm 2 provides the pseudo code for how we calculate time. We treat each step along the planned path as a step in time. Whenever we add a new state to a planner's vertex tree, we assume it will be part of the final path and we assign a step count to it, indicating the time the robot would be in that state. Line 3 of Algorithm \ref{pritime} shows how the time is predicted, by simply adding one to the previous vertex's time. Line 10 shows how we use the time value as an index to the corresponding state from the path of another already solved robot for collision detection.

%%%%%%%%%%%%%%%%%%%
% pdeudo code
\begin{algorithm}
\caption{Timing Strategy}\label{pritime}
\begin{algorithmic}[1]
\State $sto \gets Random~State$
\State $v \gets Choose~Vertex~Connection$
\State $sto.time \gets v.time + 1$
\ForAll{$X_{i}$ in $Obstacles$}
\If{$IsCollision(sto, X_{i})$}
\State return
\EndIf
\EndFor
\ForAll{$A_{i}$ in $A_{solved==true}$}
\If{$IsCollision$($sto$, $A_{i}.path$[$sto.time$]$)$}
\State return
\EndIf
\EndFor
\State $AddNewVertex(sto)$
\end{algorithmic}
\end{algorithm}
%%%%%%%%%%%%%%%%%%%

\par
Prioritized planning may use any single robot planning algorithm to plan the individual robots. We migrated 3 implementations of single robot tree based planners from our Homework 3 to work in the multi-robot framework for experimentation. Each planner builds a tree of valid random states, and uses some strategy to connect new samples to a vertex in the tree. Random picks a vertex uniformly at random from the tree. EST picks a vertex randomly using a weight heuristic to bias the choice toward vertices with fewer child nodes. RRT chooses the vertex based on shortest euclidean distance.

\subsection{Path Smoothing}
In order to explore basic path smoothing we took the path generated by the RRT algorithm and smoothed between the path vertices, although we did not include smoothing in our data collection experiments. We modeled our smoothing solution on pseudo code we found in \cite{jarvis}. We looked at the first pair(LastNode, NextNode) of vertices and if there was a valid path between them then we set NextNode to the next vertex in the path and compared them again.  When we come to a pair of vertices that don't have a valid path then we add the previous point(NextNode-1) to the path and set LastNode equal to NextNode-1 and NextNode equal to the next vertex.  
\par
This algorithm provides a little smoothing to the path found by RRT.  There are definitely places that the algorithm doesn't find the best path but that can be explained by the way the algorithm finds which two points to connect.   There are some instances where if we compared all points, then generated the path that we could find a lot smoother path.  This algorithm short circuits to the last point found before an invalid path so it can't skip over obstacles.  In Figure \ref{fig:smoothing} the two points notated by the black arrows are connected because the LastNode, the point below arrow 1, is compared to the NextNode while it walks down the path until there is an obstacle.  NextNode ends up at the second arrow.  If the algorithm was smarter it could have cut off a lot of the corner.

%%%%%%%%%%%%%%%%%%%
% scene images
\begin{figure}[ht]
\centering
{\includegraphics[width=1.5in,height=1.75in,clip,keepaspectratio]{path_smoothing.jpg}}
\caption{Limitation of path smoothing algorithm}\label{fig:smoothing}
\end{figure}
%%%%%%%%%%%%%%%%%%%

%%%%%%%%%%%%%%%%%%%%%%%%%%%%%%%%%%%%%%%%%%%%%%%%%%%%%%%%%%%%%
\section{Experimentation}
Our experimentation focused on scene configurations which expose timing conflicts among robots. We set out with the intention of demonstrating how planning order can impact the ability of the algorithm to find a solution for all robots as motivation for comparing priority functions. As we progressed in experimentation we observed that we could easily produce scenes where all priority orders would lead to failure even though the scene was solvable. Our planning experiments did not include path smoothing.
\par
This section is broken into 2 parts. First we discuss scenes where priority is a primary concern, and show how our priority function can influence choosing the correct priority. Second, we discuss scenes where factors other than priority cause the scene to be challenging for planners and suggest possible solutions presented by other researchers that should be investigated in our future research. Except where specifically noted, all of the test robots used basic RRT as the planning algorithm in conjunction with prioritized planning.

\subsection{Priority Function Results}\label{sect:priority}
These experiments are designed around a scene where priority matters. Figure \ref{fig:basic1} depicts the initial state of a scene where a Red robot is positioned below an Orange robot's goal in a tight corridor. We performed each test 10 times, giving the robots 60 seconds of processing time to find a path for both robots. Success is measured by both robots reaching the goal, if one robot fails to find the goal the test is qualified as a failure. The average and mean times count  the cumulative processing time of both robots to reach the goal, and failed tests are not included in those numbers.
\par
Figure \ref{fig:basic2} depicts the same scene after the robots have had time to attempt a solution, and Orange was planned first. In this case, Orange traveled quickly into the corridor, and since Orange has clogged the exit Red cannot plan a way out. Figure \ref{fig:basic3} depicts the same scene again after useing our average obstacle distance priority function. Because Red is packed in more tightly by the obstacles than Orange is in the initial scene, the priority function plans Red first. In this scene the corridor was blocked as Red traversed out of the corridor. This prevented Orange from finding the most direct route, but after a short detour Orange returns to the corridor later when it is clear.

%%%%%%%%%%%%%%%%%%%
% scene images
\begin{figure}[ht]
\centering
\makebox[3in]{\begin{subfigure}[p]{0.15\textwidth}
	{\includegraphics[width=1in,height=1.25in,clip,keepaspectratio]{basic_scene_a.jpg}}
	\caption{Initial Scene}
	\label{fig:basic1}
\end{subfigure}
\begin{subfigure}[p]{0.15\textwidth}
	{\includegraphics[width=1in,height=1.25in,clip,keepaspectratio]{basic_scene_b.jpg}}
	\caption{Orange First}
	\label{fig:basic2}
\end{subfigure}
\begin{subfigure}[p]{0.15\textwidth}
	{\includegraphics[width=1in,height=1.25in,clip,keepaspectratio]{basic_scene_c.jpg}}
	\caption{Red First}
	\label{fig:basic3}
\end{subfigure}}
\caption{Priority can influence solution}\label{fig:basicscenes}
\end{figure}
%%%%%%%%%%%%%%%%%%%

Table \ref{basicresults} shows the results of testing. The success ratings clearly demonstrate that priority matters in this scene, with a low success rate of 40\% when Orange planned first. The perfect success rate observed when Red planned first is paired with a relatively high time to solution. Those high times are a result of the Orange planner having to work harder since for many of the early steps the corridor would have been blocked while Red exited. The fact that choosing Red was done dynamically at runtime before the planning process began shows that average obstacle distance can be a reasonable heuristic for priortized planning in some scenes. However, we will further discuss the implications of priority in the next section, as we informally produced scenes where priority does not influence the outcome and have identified this as an area of research beyond this document.

%%%%%%%%%%%%%%%%%%%
\begin{table}[ht]
\renewcommand{\arraystretch}{1.3}
\caption{Impact of Priority}
\label{basicresults}
\centering 
\begin{tabular}{||c||c||c||c||}
\hline
\bfseries Scene & \bfseries Avg & \bfseries Mean & \bfseries Success\\
\hline\hline
\ref{fig:basic2} & 0.003170 & 0.005511 & 40\%\\
\hline
\ref{fig:basic3} & 1.715404 & 0.200755 & 100\%\\
\hline
\hline
\end{tabular}
\end{table}
%%%%%%%%%%%%%%%%%%%

% needed in second column of first page if using \IEEEpubid
%\IEEEpubidadjcol

\subsection{Importance of Timing}
While Section \ref{sect:priority} demonstrated how rearranging the order of planning can improve the planning success rate, we observed a number of scene states where priority does not resove the robot traffic jam. The scenes in Figure \ref{fig:hardscenes} show an initial state that cannot be solved easily by Basic RRT regardless of the planning order. Scenes \ref{fig:hard2} and \ref{fig:hard3} show how the corridor can be blocked by whoever plans first. We again performed comparison tests on the same starting state, but alternatvie planning orders and collected data for 10 runs on each configuration.

%%%%%%%%%%%%%%%%%%%
% scene images
\begin{figure}[ht]
\centering
\makebox[3in]{\begin{subfigure}[p]{0.15\textwidth}
	{\includegraphics[width=1in,height=1.25in,clip,keepaspectratio]{basic_hard_a.jpg}}
	\caption{Initial Scene}
	\label{fig:basic1}
\end{subfigure}
\begin{subfigure}[p]{0.15\textwidth}
	{\includegraphics[width=1in,height=1.25in,clip,keepaspectratio]{basic_hard_b.jpg}}
	\caption{Orange First}
	\label{fig:hard2}
\end{subfigure}
\begin{subfigure}[p]{0.15\textwidth}
	{\includegraphics[width=1in,height=1.25in,clip,keepaspectratio]{basic_hard_c.jpg}}
	\caption{Red First}
	\label{fig:hard3}
\end{subfigure}}
\caption{Timing can obstruct solution}\label{fig:hardscenes}
\end{figure}
%%%%%%%%%%%%%%%%%%%

%%%%%%%%%%%%%%%%%%%
\begin{table}[ht]
\renewcommand{\arraystretch}{1.3}
\caption{Limitations of Timing with Basic RRT}
\label{hardresults}
\centering 
\begin{tabular}{||c||c||c||c||}
\hline
\bfseries Scene & \bfseries Avg & \bfseries Mean & \bfseries Success\\
\hline\hline
\ref{fig:hard2} & - & - & 0\%\\
\hline
\ref{fig:hard3} & 28.338652 & 28.338652 & 10\%\\
\hline
\hline
\end{tabular}
\end{table}
%%%%%%%%%%%%%%%%%%%

Table \ref{hardresults} provides the experimental results, and shows a low success rate for either planning order. These scenes and results expose a weakness in our overall algorithm. We identify two specific causes to explain these failures. 
\par
First, Basic RRT chooses to connect points to its tree by euclidean distance. RRT tends to choose paths which are efficient in the sense that it connects points alongs paths which arrive at a new location with minimal meandering. In the case of \ref{fig:hard2}, there is nothing new and we can see that priority would be a factor similar to \ref{sect:priority}. However, in the case of \ref{fig:hard3}, the Orange robot should be able to find a path but usually cannot. This is because Orange is able to arrive at the corridor entrance too early, before Red has exited.
\par
Observing the early arrival of Orange brings us to our second explanation which is our timing function. Our timing function does not permit the robot to sit still and wait, but instead the robot wants to always march forward along its path. RRT is locked into choosing the nearest vertex, which inadvertently connects position and time, limiting RRT's ability to coordinate an alternative path.
\par
We want to identify both a smarter timing solution, and single robot planner that can take advantage of it. These observations motivate a search for alternative solutions in literature. \cite{bergprm} identifies similar challenges for prioritized planning, but suggests a prioritization function based on an underlying roadmap rather than letting each robot sample and build a path for the scene. We like the PRM approach because it is designed for multiple queries, which is exactly what a prioritized planning solution for multiple robots requires. A perfectly efficient road map could limit the robots ability to choose alternative paths, and research in \cite{bergprm} was done to identify that good roadmaps for this problem will include cycles. That paper references the earlier \cite{overcycles} which discusses strategies for deciding which cycles should be included in the graph, as a way of maintaining a balance of efficiency with flexibility.

%%%%%%%%%%%%%%%%%%%%%%%%%%%%%%%%%%%%%%%%%%%%%%%%%%%%%%%%%%%%%
\section{Challenges and Discussion}
\emph{lorem ipsum dolor} blah blah blah....

\subsection{Subsection Heading Here}
Subsection text here.

% needed in second column of first page if using \IEEEpubid
%\IEEEpubidadjcol

\subsubsection{Subsubsection Heading Here}
Subsubsection text here.

%%%%%%%%%%%%%%%%%%%%%%%%%%%%%%%%%%%%%%%%%%%%%%%%%%%%%%%%%%%%%
% An example of a floating figure using the graphicx package.
% Note that \label must occur AFTER (or within) \caption.
% For figures, \caption should occur after the \includegraphics.
% Note that IEEEtran v1.7 and later has special internal code that
% is designed to preserve the operation of \label within \caption
% even when the captionsoff option is in effect. However, because
% of issues like this, it may be the safest practice to put all your
% \label just after \caption rather than within \caption{}.
%
% Reminder: the "draftcls" or "draftclsnofoot", not "draft", class
% option should be used if it is desired that the figures are to be
% displayed while in draft mode.
%
%\begin{figure}[!t]
%\centering
%\includegraphics[width=2.5in]{myfigure}
% where an .eps filename suffix will be assumed under latex, 
% and a .pdf suffix will be assumed for pdflatex; or what has been declared
% via \DeclareGraphicsExtensions.
%\caption{Simulation Results}
%\label{fig_sim}
%\end{figure}

% Note that IEEE typically puts floats only at the top, even when this
% results in a large percentage of a column being occupied by floats.
% However, the Computer Society has been known to put floats at the bottom.


% An example of a double column floating figure using two subfigures.
% (The subfig.sty package must be loaded for this to work.)
% The subfigure \label commands are set within each subfloat command, the
% \label for the overall figure must come after \caption.
% \hfil must be used as a separator to get equal spacing.
% The subfigure.sty package works much the same way, except \subfigure is
% used instead of \subfloat.
%
%\begin{figure*}[!t]
%\centerline{\subfloat[Case I]\includegraphics[width=2.5in]{subfigcase1}%
%\label{fig_first_case}}
%\hfil
%\subfloat[Case II]{\includegraphics[width=2.5in]{subfigcase2}%
%\label{fig_second_case}}}
%\caption{Simulation results}
%\label{fig_sim}
%\end{figure*}
%
% Note that often IEEE papers with subfigures do not employ subfigure
% captions (using the optional argument to \subfloat), but instead will
% reference/describe all of them (a), (b), etc., within the main caption.


% An example of a floating table. Note that, for IEEE style tables, the 
% \caption command should come BEFORE the table. Table text will default to
% \footnotesize as IEEE normally uses this smaller font for tables.
% The \label must come after \caption as always.
%
%\begin{table}[!t]
%% increase table row spacing, adjust to taste
%\renewcommand{\arraystretch}{1.3}
% if using array.sty, it might be a good idea to tweak the value of
% \extrarowheight as needed to properly center the text within the cells
%\caption{An Example of a Table}
%\label{table_example}
%\centering
%% Some packages, such as MDW tools, offer better commands for making tables
%% than the plain LaTeX2e tabular which is used here.
%\begin{tabular}{|c||c|}
%\hline
%One & Two\\
%\hline
%Three & Four\\
%\hline
%\end{tabular}
%\end{table}


% Note that IEEE does not put floats in the very first column - or typically
% anywhere on the first page for that matter. Also, in-text middle ("here")
% positioning is not used. Most IEEE journals use top floats exclusively.
% However, Computer Society journals sometimes do use bottom floats - bear
% this in mind when choosing appropriate optional arguments for the
% figure/table environments.
% Note that, LaTeX2e, unlike IEEE journals, places footnotes above bottom
% floats. This can be corrected via the \fnbelowfloat command of the
% stfloats package.

\section{Conclusion}
We demonstrated a basic implementation of prioritized planning for multiple robots using Basic RRT as the underlying single robot planner. This algorithm works reasonable well in general cases where there is limited opportunity for begin stuck in tight corridors. We exposed some of the problem areas presented by this approach in relation to timing and the choice of the underlying planning algorithm. Research has been done by others on this problem, and additional work should be done to who how alternative approaches might solve the scenes with higher success rates.

% Can use something like this to put references on a page
% by themselves when using endfloat and the captionsoff option.
\ifCLASSOPTIONcaptionsoff
  \newpage
\fi



% trigger a \newpage just before the given reference
% number - used to balance the columns on the last page
% adjust value as needed - may need to be readjusted if
% the document is modified later
%\IEEEtriggeratref{8}
% The "triggered" command can be changed if desired:
%\IEEEtriggercmd{\enlargethispage{-5in}}

% references section

% can use a bibliography generated by BibTeX as a .bbl file
% BibTeX documentation can be easily obtained at:
% http://www.ctan.org/tex-archive/biblio/bibtex/contrib/doc/
% The IEEEtran BibTeX style support page is at:
% http://www.michaelshell.org/tex/ieeetran/bibtex/
%\bibliographystyle{IEEEtran}
% argument is your BibTeX string definitions and bibliography database(s)
%\bibliography{IEEEabrv,../bib/paper}
%
% <OR> manually copy in the resultant .bbl file
% set second argument of \begin to the number of references
% (used to reserve space for the reference number labels box)
\begin{thebibliography}{1}

\bibitem{lavalle}
S.~M.~LaValle, \emph{Planning Algorithms}\hskip 1em plus
  0.5em minus 0.4em\relax University of Illinois: Cambridge University Press, 2006, pp. 266-269.

\bibitem{slides}
A.Shehu, \emph{CS689 - Planning Motions of Robots}\hskip 1em plus
  0.5em minus 0.4em\relax George Mason University: Course Slides for Multiple Robots.
 
\bibitem{bergprm}
J.~P.~van den Berg and M.~H.~Overmars, \emph{Roadmap-Based Motion Planning in Dynamic Environments}\hskip 1em plus
  0.5em minus 0.4em\relax IEEE Transactions on Robotics, vol 21, no. 5, October 2005, pp. 885-897.

\bibitem{bergpri}
J.~P.~van den Berg and M.~H.~Overmars, \emph{Prioritized Motion Planning for Multiple Robots}\hskip 1em plus
  0.5em minus 0.4em\relax IEEE/RSJ International Conference on Intelligent Robots and Systems, 2005, pp. 430-435.

\bibitem{overcycles}
D.~Nieuvenhuisen and M.~H.~Overmars, \emph{Useful Cycles in Probabilistic Roadmap Graphs}\hskip 1em plus
  0.5em minus 0.4em\relax IEEE International Conference on Robotics and Automation, vol 1, 2004, pp. 446-452.
 
\bibitem{jarvis}
Z.~Deak and J.~Jarvis, \emph{Robotic Path Planning using Rapidly exploring Random Trees},\hskip 1em plus 0.5em minus 0.4em\relax : Intelligent Robotics Research Center - Monash University.
  
\end{thebibliography}
\end{document}


